%   MSc Business Analytics Dissertation/Practicum
%
%   Title:     Enhancing Credit Analysis and Assessment using GeoSpatial Techniques
%   Author(s): Deepak Gupta and Shruti Goyal
%
%   This file is the top level file for a dissertation.  It contains:
%   - usepackage commands for any packages required
%   - document-wide parameter settings such as textwidth, etc
%   - The text of small items of the frontmatter (title, abstract, etc.) and backmatter
%
%   Chapter 1: Introduction and basic definitions
%   Chapter 2: Business background and problem
%   Chapter 3: Literature Review
%   Chapter 4: Methodology
%   Chapter 5: Results
%   Chapter 6: Discussion
%   Chapter 7: Conclusions
%
%   Change Control:
%   When     Who   Ver  What
%   -------  ----  ---  --------------------------------------------------------------
%   01Sep10  SMcG  0.0  Developed MSc Business Analytics Dissertation LaTeX template
%   02Sep16  JMcD  0.1  Updated template
%   01Apr17  XX    0.2  Begun dissertation proper
%   02Apr17  YY    0.3  Added ....
%

%% To LaTeX (compile) your whole document, run these \textbf{Response Variable:} DefaultedLoans\\

%% the command line on Linux or Mac OS X but you can run them from inside WinEdt or whatever you use):
%% latex  yourfilename
%% bibtex yourfilename  % NB this is the name of your LaTeX file, not your BibTeX database file!!
%% latex  yourfilename
%% latex  yourfilename
%%
%% Note: you don't need the .tex extension of yourfilename.tex (in fact giving it will confuse bibtex,
%% though latex will understand it).  If you stick with the name thesis.tex for this file, you would use
%% latex thesis etc
%%
%% There are two approaches to where to put the supporting files (LaTeX style (package) files, etc).  Either:
%% 1. Put them all in the same directory as this file (quick and dirty); or
%% 2. Many style (package) files such as natbib.sty, lineno.sty may already be installed on your system, but if
%%    not, just download them e.g., from ctan.org, and copy them to YOURBASEDIRECTORY/texmf/tex/latex
%%    Copy the BibTeX style file mscBA.bst to YOURBASEDIRECTORY/texmf/bibtex/bst
%% Approach 2 will make these files visible to any other LaTeX docs you produce.  Here, YOURBASEDIRECTORY is
%% typically your home directory.

\documentclass[a4paper,12pt,bibtotoc,notitlepage,oneside]{book}
% remove ``oneside'' when printing the final version since that
% will have plenty of blank pages already!

% the bibtotoc may be needed to make your references appear in the Contents, also liststotoc etc

% Preamble

% LaTeX Packages to use
\usepackage{makeidx}
%\usepackage{showidx} % for drafts: shows in margin what will be indexed
\usepackage{amsthm,amsmath,amsfonts,amssymb,latexsym,graphicx,appendix,subfigure}
% the ams* packages provide the (very useful) LaTeX extensions from the AMS (Amer. Math. Soc.)
% latexsym provides some useful extra symbols
% graphicx gives control over image files, e.g., width, etc
% appendix gives more fine-tuned control over the appendices
% subfigure lets you put multiple figures (and tables) in a figure environment, so you can refer to them individually, e.g., Figure~6(b).
\usepackage{algorithm2e} % handles algorithms in a nice way, akin to tables and figures
\usepackage[subfigure]{tocloft}  % package allows you to control the design of table of contents, figures and tables. use subfigure for compatibility -- jmmcd
\usepackage{natbib}   % gives more control over references and style of citation.  Has a lot of options, e.g., can use the [sectionbib] option to have multiple bibliographies in my document, listed in sections rather than chapters. Need this package for Chicago Manual of Style BibTeX referencing and the MScBA derived style
\usepackage{lineno}   % useful for early draft, so a reviewer can list by pageno+lineno where to make changes
% Packages you may like to use
% \usepackage{fancyhdr}  % more control over contents of page headers and footers for documentclass book
% \usepackage{psfrag}  % if using eps files for images, this package lets you replace text in those images, so you can have the same font type and size in both text and images.
% \usepackage{layout} % prints an overview of all the page margins, if you include \layout
%\usepackage[sectionbib]{natbib} if using multiple bibliography files
%\usepackage{chapterbib}
%\usepackage{doublespace} % for double spacing, e.g. initial versions for correcting
%\usepackage[nottoc]{tocbibind} % puts almost everything in the Table of Contents

% include your own macro / declaration files, if any
\usepackage{booktabs}
\usepackage{graphicx}
%   Macros
%   Esp. Theorem styles for amsart/amsbook document classes (currently enabled)
%
%   Sean McGarraghy
%
%   Change Control:
%   Date      Ver  Reason
%   --------  ---  --------------------------------------------------------------
%   12/05/98  0.1  Begun
%

\newcommand*{\Ax}  {Axiom~}
\newcommand*{\Df}  {Definition~}
\newcommand*{\Th}  {Theorem~}
\newcommand*{\Co}  {Corollary~}
\renewcommand*{\Pr}{Proposition~}
\newcommand*{\Lm}  {Lemma~}
\newcommand*{\Rk}  {Remark~}
\newcommand*{\Ex}  {Example~}
\newcommand*{\Exe} {Exercise~}
\newcommand*{\Eq}  {Equation~}
\newcommand*{\dfn} {definition}

\theoremstyle{plain}                 % Default

\newtheorem{thm}{Theorem}[chapter]
\newtheorem{lemma}[thm]{Lemma}
\newtheorem{lem}[thm]{Lemma}
\newtheorem{propn}[thm]{Proposition}
\newtheorem{pr}[thm]{Proposition}
\newtheorem{prp}[thm]{Propn.}
\newtheorem{cor}[thm]{Corollary}
\newtheorem{fact}[thm]{Fact}
\newtheorem{conj}[thm]{Conjecture}
\newtheorem{claim}{Claim}[]

\theoremstyle{definition}

\newtheorem{dm}{}  % Dummy theorem name for numbering only, independent of section
\newtheorem{dmy}[thm]{}  % Dummy theorem name for numbering only
\newtheorem{defn}[thm]{Definition}
\newtheorem{ex}[thm]{Example}
\newtheorem{exe}[thm]{Exercise}

\theoremstyle{remark}

\newtheorem{rmk}[thm]{Remark}
\newtheorem{notn}[thm]{Notation}
\newtheorem{note}[thm]{Note}
\newtheorem{cmt}[thm]{Comment}
\newtheorem{case}[thm]{Case}
\newtheorem{intr1}[thm]{Introduction}
\newtheorem{intro}[thm]{Introduction}

\newenvironment{pf}{\vspace{-4pt}\noindent\begin{proof}}{\end{proof}\vspace{2pt}}




%   General LaTeX Declarations (try -adobe-courier-medium-r-normal--18-180-75-75-m-110-iso8859-2)
%
%   Sean McGarraghy
%
%   Change Control:
%   Date      Ver  Reason
%   --------  ---  --------------------------------------------------------------
%   12/02/98  0.1  Begun
%

% Where to hyphenate words LaTeX doesn't know

\hyphenation{neigh-bour-hood neigh-bour-hoods}
\hyphenation{Bel-field}

% Short commands for italic, bold etc fonts

\newcommand*{\tr}[1]{\textrm{#1}}
\newcommand*{\ti}[1]{\textit{#1}}
\newcommand*{\tb}[1]{\textbf{#1}}
\newcommand*{\ts}[1]{\textsl{#1}}
\newcommand*{\tc}[1]{\textsc{#1}}
\newcommand*{\mr}[1]{\mathrm{#1}}
\newcommand*{\mb}[1]{\mathbf{#1}}
\newcommand*{\tth}[1]{{\slshape #1}}
\newcommand*{\df}[1]{\textbf{#1}}      % put the thing being defined in bold

\newcommand*{\bs}{$\backslash$}
\newcommand*{\ol}[1]{\overline{#1}}
\newcommand*{\cnj}[1]{\overline{#1}}
\newcommand*{\nth}[1]{\ensuremath{{#1}^{\textrm{th}}}}% superscripted `th' for counting
\newcommand*{\nst}[1]{\ensuremath{{#1}^{\textrm{st}}}}% superscripted `st' for counting
\newcommand*{\nnd}[1]{\ensuremath{{#1}^{\textrm{nd}}}}% superscripted `nd' for counting
\newcommand*{\nrd}[1]{\ensuremath{{#1}^{\textrm{rd}}}}% superscripted `rd' for counting

% commonly occurring text in Maths documents

\newcommand*{\eg}  {e.\,g.~}
\newcommand*{\ie}  {i.\,e.~}
\newcommand*{\ff}  {\textrm{if and only if }}
\newcommand*{\A}   {\textrm{ for all }}
\newcommand*{\all} {\textrm{ for all }}
\renewcommand*{\AA}{\ensuremath{\forall~}}
\newcommand*{\E}   {\textrm{ there exists }}
\newcommand*{\EE}  {\ensuremath{\exists~}}
\newcommand*{\st}  {\textrm{ such that }}
\newcommand*{\sta} {\textrm{ such that, for all }}
\newcommand*{\tfae}{the following are equivalent}
\newcommand*{\Tfae}{The following are equivalent}
\newcommand*{\wrt} {with respect to }
\newcommand*{\wolog}{without loss of generality}
\newcommand*{\Wolog}{Without loss of generality}
\newcommand*{\ftsoc}{for the sake of contradiction}
\newcommand*{\resp}{respectively}
\newcommand*{\RHS} {right hand side}
\newcommand*{\LHS} {left hand side}
\newcommand*{\Pf}  {\noindent\tb{Proof: }}            % for backwards compatibility only
\newcommand*{\eop} {~~\vrule height 5 pt width 5 pt depth 0 pt\relax}
\newcommand*{\QED} {\mbox{}\hspace{\fill}\eop}

\newcommand*{\fn}  {function}
\newcommand*{\vsp} {vector space}

\newcommand*{\ind} {independent}
\newcommand*{\dep} {dependent}
\newcommand*{\lcmb}{linear combination}
\newcommand*{\lind}{linearly independent}
\newcommand*{\ldep}{linearly dependent}
\newcommand*{\di}  {dimension}
\newcommand*{\fd}  {finite-dimension}
\newcommand*{\cp}  {characteristic polynomial}
\newcommand*{\de}  {determinant}
\newcommand*{\evl} {eigenvalue}
\newcommand*{\evc} {eigenvector}
\newcommand*{\sle} {system of linear equations}
\newcommand*{\ERO} {elementary row operation}
\newcommand*{\ECO} {elementary column operation}
\newcommand*{\repn}{representation}

\newcommand*{\UCD} {University College Dublin} 

\newcommand*{\itema}[1][a]{\item[(\emph{#1})]} 
\newcommand*{\ita}[1][a]{(\emph{#1})} 

% Math Roman abbreviations e.g. Gal(), Aut() etc.

\newcommand*{\len}{\ensuremath{\operatorname{length}}}% 
\newcommand*{\rank}{\ensuremath{\operatorname{rank}}} % RANK of matrix/linear map
\newcommand*{\nll}{\ensuremath{\operatorname{nullity}}} % NULLity of linear map
\newcommand*{\diag}{\ensuremath{\operatorname{diag}}} % DIAGonal matrix
\newcommand*{\cond}{\ensuremath{\operatorname{cond}}} % CONDition number
\newcommand*{\sign}{\ensuremath{\operatorname{sign}}} % SIGNature
\newcommand*{\sgn} {\ensuremath{\operatorname{sgn}}}  % SiGN of permutation
\newcommand*{\imp}{\ensuremath{\mathbin\Rightarrow}}  % IMPlies sign
\renewcommand*{\Im}{\ensuremath{\operatorname{Im}}}   % IMage (of a map)
\newcommand*{\zv} {\ensuremath{\mathbf{0}}}           % Zero vector
\newcommand*{\zs}[1]{\ensuremath{{#1}^{-1}(0)}}       % Zero Set of function #1
\renewcommand*{\i}[1]{\ensuremath{{#1}^{-1}}}         % Inverse of geezer #1

% spacing/heading commands

\newcommand*{\cpb}{\vspace{\fill}\pagebreak}          % Clean PageBreak inside paragraph
\newcommand*{\cb}{\hspace*{\fill}\\*[\medskipamount]} % Clean Break inside paragraph
\newcommand*{\cbb}{\hspace*{\fill}\\[\medskipamount]} % Clean Break, may have page Break
\newcommand*{\cbs}{\hspace*{\fill}\\*[\smallskipamount]} % Small Clean Creak inside paragraph
\newcommand*{\cbbs}{\hspace*{\fill}\\[\smallskipamount]} % Small Clean Creak, may have page break
\newcommand*{\cbl}{\hspace*{\fill}\\*}                % No gap Clean Break inside paragraph
\newcommand*{\cbbl}{\hspace*{\fill}\\}                % No gap Clean Break, may have page break

\newcommand*{\qt}[1]{\ensuremath{\quad\text{#1}\quad}}
\newcommand*{\qqt}[1]{\ensuremath{\qquad\text{#1}\qquad}}

\newcommand*{\subhead}[1]{  
            \begin{center}
            \tb{#1}
            \end{center}\\*}


% Various inclusion, ideal, equivalence, mapping symbols

%\newcommand*{\squig}{\sim\!\sim\!\rightarrow}  % old definition
\newcommand*{\eqv} {\ensuremath{\mathbin\Leftrightarrow}}  % EQuiValence sign
\newcommand*{\app} {\ensuremath{\rightarrow}}          % APProaches/tends to (limit)
\newcommand*{\map} {\ensuremath{\longrightarrow}}      % set MAPs to (long arrow)
\newcommand*{\tmap}[1]{\ensuremath{
                      \stackrel{\ssk{#1}}{\map} }}    % scriptsize #1 atop ---> arrow
\newcommand*{\mpt}{\ensuremath{\longmapsto}}          % element maps to (long arrow with bar)
\newcommand*{\tmpt}[1]{\ensuremath{
                               \stackrel{#1}{\mpt}}}  % scriptsize #1 atop |--> arrow
\newcommand*{\cng}{\ensuremath{\equiv}}               % CoNGruent to 
\newcommand*{\rel}{\ensuremath{\sim}}                 % single tilde RELation sign
\newcommand*{\nul}{\ensuremath{\emptyset}}            % NULl set 
\newcommand*{\su} {\ensuremath{\subset}}              % SUbset of
\newcommand*{\se} {\ensuremath{\subseteq}}            % Subset of/Equal to
\newcommand*{\sne}{\ensuremath{\subsetneqq}}          % Subset of but Not Equal to
\newcommand*{\sm} {\ensuremath{\smallsetminus}}       % set difference (Set Minus)
\newcommand*{\la} {\ensuremath{\langle}}              % Left Angle bracket
\newcommand*{\ra} {\ensuremath{\rangle}}              % Right Angle bracket
\newcommand*{\du} {\ensuremath{\dot{\cup}}}           % Disjoint Union (dot over cup sign)
\newcommand*{\rt}{\right}
\newcommand*{\lt}{\left}
\newcommand*{\oo}{\ensuremath{\infty}}                % oo = infinity
\newcommand*{\fp}[3]{\ensuremath{{#1}:{#2}\map{#3}}}  % Function/maP f:A--->B
\newcommand*{\ma}[3]{\ensuremath{{#1}:\R^{#2}\map\R^{#3}}} % MAp f:R^n--->R^m
\newcommand*{\lma}[4][]{linear map{#1} \ma{#2}{#3}{#4}} % linear map(s) f:R^n--->R^m
\newcommand*{\cpd}[1][A]{\ensuremath{\det({#1}-\l I)}} % char poly eg det(A-lI)

% Logic and proof
\newcommand*{\LT} {\texttt{T}}
\newcommand*{\LF} {\texttt{F}}
\newcommand*{\NOT}{\mathop\mathrm{not}}
\newcommand*{\AND}{\mathbin\mathrm{and}}
\newcommand*{\OR} {\mathbin\mathrm{or}}

% Binomial coefficient
\newcommand*{\bin}[2]{\ensuremath{\binom{#1}{#2}}}

% Binomial coefficient (textstyle i.e. Small font)
\newcommand*{\sbin}[2]{\ensuremath{\tbinom{#1}{#2}}}

% Binomial coefficient (displaystyle i.e. Big font)
\newcommand*{\bbin}[2]{\ensuremath{\dbinom{#1}{#2}}}

% derivatives
\newcommand*{\dd}[1]{\ensuremath{\displaystyle{\frac{d}{d{#1}}}}}

\newcommand*{\polyn}[3][n]{\ensuremath{{#3}_0+{#3}_1{#2}+\cdots+{#3}_{#1}{#2}^{#1}}}

% MATRICES

% two commonly used building blocks for matrices

% identity matrix
\newcommand*{\idmat}{\ensuremath{ 
1      & 0      & \cdots & 0       \\
0      & 1      & \cdots & 0       \\
\vdots & \vdots & \ddots & \vdots  \\
0      & 0      & \cdots & 1          } }

% matrix consisting of zeros
\newcommand*{\zeromat}{\ensuremath{ 
0      & 0      & \cdots & 0       \\
\vdots & \vdots & \ddots & \vdots  \\
0      & 0      & \cdots & 0          } }

% matrix consisting of asterisks
\newcommand*{\astmat}{\ensuremath{ 
*      & *      & \cdots & *       \\
\vdots & \vdots & \ddots & \vdots  \\
*      & *      & \cdots & *          } }

% Greek letters - abbreviations

\renewcommand*{\a}{\ensuremath{\alpha}}
\renewcommand*{\b}{\ensuremath{\beta}}
\newcommand*{\g}{\ensuremath{\gamma}}
\renewcommand*{\d}{\ensuremath{\delta}}
\newcommand*{\e}{\ensuremath{\varepsilon}}
\newcommand*{\z}{\ensuremath{\zeta}}
\newcommand*{\h}{\ensuremath{\eta}}
\renewcommand*{\th}{\ensuremath{\vartheta}}
%\newcommand*{\io}{\ensuremath{\iota}}
\renewcommand*{\k}{\ensuremath{\kappa}}
\renewcommand*{\l}{\ensuremath{\lambda}}
\newcommand*{\m}{\ensuremath{\mu}}
\newcommand*{\n}{\ensuremath{\nu}}
\renewcommand*{\r}{\ensuremath{\rho}}
\newcommand*{\s}{\ensuremath{\sigma}}
\renewcommand*{\t}{\ensuremath{\tau}}
\renewcommand*{\u}{\ensuremath{\upsilon}}
\newcommand*{\f}{\ensuremath{\varphi}}
\renewcommand*{\c}{\ensuremath{\chi}}
\newcommand*{\x}{\ensuremath{\xi}}
\newcommand*{\p}{\ensuremath{\psi}}
\renewcommand*{\o}{\ensuremath{\omega}}

\newcommand*{\G}{\ensuremath{\Gamma}}
\newcommand*{\D}{\ensuremath{\Delta}}
\newcommand*{\T}{\ensuremath{\Vartheta}}
\renewcommand*{\L}{\ensuremath{\Lambda}}
\renewcommand*{\L}{\ensuremath{{\textstyle{\bigwedge}}}}
%\newcommand*{\Si}{\ensuremath{\Sigma}}
\newcommand*{\U}{\ensuremath{\Upsilon}}
\newcommand*{\F}{\ensuremath{\Phi}}
\renewcommand*{\P}{\ensuremath{\Psi}}
\renewcommand*{\O}{\ensuremath{\Omega}}


% Math mode blackboard and calligraphic (script) letters

\newcommand*{\N}{\ensuremath{\mathbb{N}}}
\newcommand*{\Z}{\ensuremath{\mathbb{Z}}}
\newcommand*{\Q}{\ensuremath{\mathbb{Q}}}
\newcommand*{\R}{\ensuremath{\mathbb{R}}}
\newcommand*{\C}{\ensuremath{\mathbb{C}}}
\renewcommand*{\H}{\ensuremath{\mathbb{H}}}
\newcommand*{\FF}{\ensuremath{\mathbb{F}}}

\newcommand*{\sA}{\ensuremath{\mathcal{A}}}
\newcommand*{\sB}{\ensuremath{\mathcal{B}}}
\newcommand*{\sC}{\ensuremath{\mathcal{C}}}
\newcommand*{\sD}{\ensuremath{\mathcal{D}}}
\newcommand*{\sE}{\ensuremath{\mathcal{E}}}
\newcommand*{\sF}{\ensuremath{\mathcal{F}}}
\newcommand*{\sG}{\ensuremath{\mathcal{G}}}
\newcommand*{\sH}{\ensuremath{\mathcal{H}}}
\newcommand*{\sI}{\ensuremath{\mathcal{I}}}
\newcommand*{\sJ}{\ensuremath{\mathcal{J}}}
\newcommand*{\sK}{\ensuremath{\mathcal{K}}}
\newcommand*{\sL}{\ensuremath{\mathcal{L}}}
\newcommand*{\sM}{\ensuremath{\mathcal{M}}}
\newcommand*{\sN}{\ensuremath{\mathcal{N}}}
\newcommand*{\sO}{\ensuremath{\mathcal{O}}}
\newcommand*{\sP}{\ensuremath{\mathcal{P}}}
\newcommand*{\sQ}{\ensuremath{\mathcal{Q}}}
\newcommand*{\sR}{\ensuremath{\mathcal{R}}}
\newcommand*{\sS}{\ensuremath{\mathcal{S}}}
\newcommand*{\sT}{\ensuremath{\mathcal{T}}}
\newcommand*{\sU}{\ensuremath{\mathcal{U}}}
\newcommand*{\sV}{\ensuremath{\mathcal{V}}}
\newcommand*{\sW}{\ensuremath{\mathcal{W}}}
\newcommand*{\sX}{\ensuremath{\mathcal{X}}}
\newcommand*{\sY}{\ensuremath{\mathcal{Y}}}
\newcommand*{\sZ}{\ensuremath{\mathcal{Z}}}


\linespread{1.25}         % spacing between lines
\parindent0pt             % no paragraph indentation
\parskip\smallskipamount  % small space between paragraphs

%\setlength{ \topmargin      }{  -1cm }
%\setlength{ \textheight     }{  24cm }
\setlength{ \textwidth      }{  14cm }
\setlength{ \oddsidemargin  }{ 1.0cm }
\setlength{ \evensidemargin }{ 1.0cm }

\numberwithin{equation}{section}  % alter this if desired
\numberwithin{figure}{chapter}    % alter this if desired

\pagestyle{plain} % just puts a page number at the bottom of each page
% If you comment out the last line's \pagestyle command, you can control the page headers as follows:
%\def\leftmark{\textsc{YourText1}}  % text to appear in header of left-hand (even numbered) pages, e.g., Chapter Title
%\def\rightmark{\textsc{YourText2}}  % text to appear in header of right-hand (odd numbered) pages, e.g., Thesis Title
%\def\leftmark{\textsc{}}  % gives empty header on left-hand (even numbered) pages
%\def\rightmark{\textsc{}}  % gives empty header on right-hand (odd numbered) pages

% If, in the output, if you find the width of numbers in the Contents, List of Figures, etc is too great
% (e.g. too close to the heading names) uncomment the following
%\setlength{\cftfignumwidth}{3em} % this allows you to control width of numbering in List of Figures
                                 % 4em, 5em etc would be even wider; you can also hardcode 2cm or similar
                                 % but units of em (width of a capital M) or ex (height of a small x) are
                                 % preferred since they vary with the font type / size used

% comment this out for your final version (where you don't want line numbers)
%\linenumbers

\makeindex % generates the index entries (unsorted: need to run makeindex from the command line to sort them)


% Use the following to only include certain chapters (it will speed up compilation of the part you
% are currently working on, should the dissertation as a whole take excessively long to compile).
\includeonly{%
preface%
,
ch1%
,%
ch2%
,%
ch3
,%
ch4%
,
ch5%
,
ch6%
,
ch7%
,
appendix1%
,
appendix2%
,
glossary%
,
notation
}


%% Lay out your thesis as follows (standard book form: some of these are optional):

%% \begin{enumerate}
%% \item Front matter (preliminaries: roman page numbering, i, ii, iii, iv,\ldots)
%%   \begin{enumerate}
%%   \itema{} Title Page, giving (on separate lines, in this order):
%%     \begin{enumerate}
%%     \itema[i] Title
%%     \itema[ii] Author(s), listing previous degree(s), e.g.,\ Joe Bloggs, B.E.
%%     \itema[iii] A thesis submitted to \UCD\ in part fulfilment of the requirements of the degree of
%%           M.Sc.~in Business Analytics
%%     \itema[iv] Michael Smurfit Graduate School of Business
%%     \itema[v] Date (\emph{Month, Year}, e.g.,\ \emph{September, 2010})
%%     \itema[vi] Supervisor(s): Xxx Xxxxx (and Yyy Yyyyy)
%%     \itema[vii] Head of School: Professor Ciar\'an O'h\'Ogartaigh
%%  \end{enumerate}
%%   \itema[b] Dedication
%%   \itema[c] Table of Contents
%%   \itema[d] List of Figures (Illustrations)
%%   \itema[e] List of Tables
%%   \itema[f] List of Algorithms (if any)
%%   \itema[g] Foreword (if any: may precede table of contents)
%%   \itema[h] Preface (including authors' signatures: may precede table of contents)
%%   \itema[j] Acknowledgements (including permissions and other credits)
%%   \itema[k] Abstract
%%   \itema[l] List of important abbreviations (including special fonts, e.g.,\ {\tt fixed width} for program code)
%%  \end{enumerate}
%% \item Body or ``Main Matter'' (the text: arabic page numbering, 1, 2, 3,\ldots)
%%   \begin{enumerate}
%%   \itema{} Chapters of your dissertation, including, but not limited to:
%%     \begin{enumerate}
%% \itema[i] Introduction
%% \itema[ii] Literature Review
%% \itema[iii] Methodology
%% \itema[iv] Results
%% \itema[v] Analysis/Discussion
%% \itema[vi] Conclusions
%%    \end{enumerate}
%%   \itema[b] Epilogue (if any)
%%   \itema[c] Afterword (if any)
%%   \end{enumerate}
%% \item Back matter (loose ends: continue arabic page numbering)
%%   \begin{enumerate}
%%   \itema{} Appendices (if any: may include program code if not too long)
%%   \itema[b] Endnotes
%%   \itema[c] Glossary of terms
%%   \itema[d] Bibliography (References)
%%   \itema[e] List of contributors
%%   \itema[f] Notation Index / List of Symbols
%%   \itema[g] Index
%% \end{enumerate}
%% \end{enumerate}


\begin{document}

\frontmatter % This is the material before chapter 1, page numbering in Roman numerals ii, iii, iv, ...

\title%[Abbreviated Title for Header]
    {\bf Enhancing Credit Analysis and Assessment using Geo Spatial Techniques\\}

\author{Deepak Kumar Gupta, B.Tech. Computer Science \\ Shruti Goyal, B.Tech. Instrumentation and Control}  % change to suit!
\date{}
\maketitle % this outputs the title using the \title and \author above

{\normalsize
\vfill
\begin{center}
% Alternative for PhD
%\textup{A thesis submitted to University College Dublin in part fulfilment
%of the requirements of the degree of Philosophi\ae~Doctor}
\textup{A Practicum submitted to University College Dublin in part fulfilment
of the requirements of the degree of M.Sc.~in Business Analytics}
\end{center}
\vfill
\begin{center}
Michael Smurfit Graduate School of Business,
University College Dublin
\end{center}
\vfill
\begin{center}
\textit{September, 2017}
\end{center}
\vfill
\begin{center}
\textup{Supervisors: Dr. Peter Keenan, UCD \\ Selwyn Hearns, KPMG IRM Audit}
\end{center}
\vfill
\begin{center}
\textup{Head of School: Professor Ciar\'an \'O h\'Ogartaigh}
\end{center}
\vfill
}

\thispagestyle{empty} % it is conventional to have no page number on the title page (page i)

\clearpage % inserts a page break


% Subsequent distinct parts of the document begin with a \chapter*{} command: this produces a chapter with
% no number.  Note that in a book or amsbook document class, each chapter starts on a new odd-numbered page.


%% Dedication
\chapter*{Dedication}
To our freinds and family for their support and encouragement.


%% Table of Contents
% If you would like the word ``Chapter '' to appear before the chapter number in your Contents, uncomment:
% \renewcommand{\cftchapfont}{Chapter }

\cleardoublepage
\tableofcontents
%\renewcommand{\cftchapfont}{}

% Note: to show subfigures, subtables (if any) in the appropriate list below, uncomment the following command
% \setcounter{lofdepth}{2}

% If you find your Table and/or Figure captions are taking up too much space in the Lists below, use the
% following in the appropriate Table and Figure environments in your chapters:
%\caption[Short Caption]{An Excessively Long and Descriptive Caption, which is probably too long in any case\ldots}
% Then the short caption appears in the list, while the long one appears in the text under your Table/Figure.

% below, the \addcontentsline{toc}{section}{List of figures} etc ensures these lists appear in the contents

% similarly to the \renewcommand{\cftchapfont}{Chapter } above, you can have the words Table or Figure appear
% in front of every entry in the appropriate list by uncommenting as approapriate:
%\renewcommand{\cftfigfont}{Figure }
%\renewcommand{\cfttabfont}{Table }

\clearpage

%% List of Figures (Illustrations)
\listoffigures\addcontentsline{toc}{chapter}{List of figures}

\clearpage
%% List of Tables
\listoftables\addcontentsline{toc}{chapter}{List of tables}

%% List of Algorithms (if any): need to \usepackage{algorithm2e} for nice algorithm layout and this command
\listofalgorithms\addcontentsline{toc}{chapter}{List of algorithms}

%% modify titles with
%% \renewcommand{\cftXtitlefont}{\Huge\itshape}
%% Replace the X with either “toc”, “lof” or “lot” and use any font size you like.
%% Available font sizes are: \tiny, \scriptsize, \footnotesize, \small, \normalsize, \large,
%% \Large, \LARGE, \huge and \Huge.



%% Preface (including authors' signatures: may precede table of contents)
%   MSc Business Analytics Dissertation
%
%   Title:     Aaa Bbbbbbb Cccccccccc
%   Author(s): Xxxxxx Xxxxxxxxx and Yyy Yyyyyyyyy
%
%   Preface
%
%   Change Control:
%   When     Who   Ver  What
%   -------  ----  ---  --------------------------------------------------------------
%   11Feb11  AB    0.1  Begun
%

\chapter*{Preface}

% Some people like to put in a little quote (Knuth in particular loves this)

\begin{quote}
\noindent\textit{Men occasionally stumble over the truth, but most of them pick themselves up and
hurry off as if nothing had happened.}

\hspace{2cm}--- Winston Churchill
\end{quote}

Much of the front matter is optional. In particular, include things like a Dedication, List of Figures, List of Tables, List of Algorithms, only if there are enough of them to justify it and it would help the reader.

Don't include both a Foreword and a Preface since they perform similar roles.

The same goes for appendices, index, glossary, list of notation and terms at the end. Include if they would help the reader.

But always, of course, include the Bibliography.

\vspace{2em}

University College Dublin \hfill Xxxxx Xxxxxxx \\
\today \hfill Yyy Yyyyyyyyy



%% Acknowledgements (including permissions and other credits).  Modify as appropriate

\chapter*{Acknowledgements}
We would like to express our thankfulness to Dr Peter Keenan and Dr James McDermott for constant support and providing answers all our queries related to this work.\\

We also like to thank KPMG, Ireland for sponsoring this project. Mr Selwyn Hearnes, Partner and Mr James Fitzpatrick from KPMG IRM Audit team for their valuable knowledge and support for carrying out this research work.


%% Abstract

%\begin{abstract}
\chapter*{Abstract/Executive Summary}

%%   List of important abbreviations (including special fonts, e.g.,\ {\tt fixed width} for program code)

% Bulk of thesis. line numbers in arabic numerals, 1, 2, 3, ...
\mainmatter

\linespread{1.3} % adjust line spacing (if desired)

% \include all your chapters.  This automatically puts the material on a new page.  Recall that \chapter{} and
% \chapter*{} start a new chapter on a new odd-numbered page, but won't put extra empty pages along with \include
%   MSc Business Analytics Dissertation
%
%   Title:     Aaa Bbbbbbb Cccccccccc
%   Author(s): Xxxxxx Xxxxxxxxx and Yyy Yyyyyyyyy
%
%   Chapter 1: Introduction and basic definitions
%
%   Change Control:
%   When     Who   Ver  What
%   -------  ----  ---  --------------------------------------------------------------
%   11Feb11  AB    0.1  Begun
%

\chapter{Introduction}\label{C.intro}
One of the key activities of banking and financial institutions that enhance their quality and financial system, correct handling, and management of liabilities. The performance of those tasks is very crucial for country's economic development, that Irish government witnessed as Irish property bubble that happened in Celtic Tiger period (late 1990 - 2007). While assessing credit risk, it is essential to validate the accuracy and reliability of credit scores or credit rating for all participants. So, How do banks identify a default event: 1. Nonrepayment of the debt to the bank, 2. Repayment is due for more than 90days.\\

This work will discuss predictive models for enhancement in credit analysis and assessment of residential mortgages registered in Ireland using geospatial locations. There have been many studies and researches on how to assess and analyze credit scoring or credit risk, but very few studies are present that describes assessment using geospatial data. This project will demonstrate how geospatial techniques can be used to enhance further credit analyses that empowers banks and financial institutions to take the much better decision on an application. This project will present a predictive model that predicts the probability of default and an interactive visualization highly focused on geospatial locations of residences registered in Ireland and bank's branch locations. The purpose of this visualization is to support decision maker to take a more efficient decision whether to provide loan on a particular house mortgage or not with the use of predicted probability of default. Models for Credit analysis are developed with the use of decision trees using CART algorithm and logistic regression for binary response (dependent) variables. While building models, potential variables were selected based on Information Value statistics. Credibility and quality of the models were evaluated using approaches such as GINI statistics, prediction accuracy, and ROC (Receiver Operating Characteristic) curve.\\

Credit assessment and analysis plays a crucial role in determining the financial strength of businesses and risk estimation that are associated with credit. Following are the primary purposes of assessment of credit:
\begin{enumerate}
\item Helps to keep track of the economy (macro economic perception) 
\item Analyses and ensures stability of financial market (macro prudential perspective)
\item Assessment of quality of collateral/mortgage (monetary policy)
\end{enumerate}

\section{Assumptions \& Challenges}

KPMG provided made up data due to a confidentiality agreement with their client. Data is generated from pre defined formulas that made data look like original real life data, but it could not cover all possible real life scenarios. For example - Data only considers that an applicant will default if it has a credit rating of 5 but data did not consider the situation that a claimant may default if he\/she has a credit score of 2,3,4 and even 1 in some cases. This case depicts a constraint of given data over real life data.\\

Below is a list of assumptions undertaken during the process of practicum:
\begin{enumerate}
\item Property prices have been considered as provided in the data by KPMG; there is no consideration of any time frame. For example, the date when property valuation was done. 
\item Geospatial data such as address latitude and address longitude is assumed to depict geospatial location property correctly.
\item A property is considered as a whole, some apartments and number of floors are ignored. What latitude and longitude of a house consist of 2 floors are same.
\item Dimensions of house and size of the house(number of rooms, bathrooms, lawn, etc.) are not considered during model development.
\item This project only focuses on residential properties, not on commercial properties. 
\item This project did not consider factors such as neighborhood, amenities, and demographics which affects the property price in the market. However, factors such as location, average price have been considered for predicting the probability of default.
\end{enumerate}

\section{Outline}
Below is the flow of the practicum which will give a brief description of each chapter:
\begin{itemize}
\item Business Background \\ This chapter describes business need and contributions in detail. It will explain how this project will contribute towards banks and financial institutions businesses. 
\item Literature Review \\ Chapter 3 presents an in-depth study of academic contributions achieved in the field of credit analysis, geospatial techniques, and data visualization. This section will explain in detail what is credit scoring and what methods have been used in the past to enhance assessment of credit. It will show a comparison between traditional systems and credit scoring along with algorithms to build a model for predicting the probability of default. Later, it will describe geospatial techniques and data visualization techniques.
\item Methodology \\ Chapter 4 will give a detailed explanation of steps and tools that have been used to successfully conduct this project and how different tools have been integrated together.
\item Results \\ Chapter 5 explains the output generated from the methods and algorithms described in the sections mentioned above. It will describe the graphs and images that hold uttermost importance and are relevant to the business need along with Tableau dashboards.
\item Discussion \\ This chapter will discuss data limitations and practicality of the models developed that correctly answers business questions. 
\item Conclusion and Future Work \\ This chapter will conclude the outcome of the practicum along with the improvements and future scope of the project.
\end{itemize}

%   MSc Business Analytics Dissertation
%
%   Title:     Aaa Bbbbbbb Cccccccccc
%   Author(s): Xxxxxx Xxxxxxxxx and Yyy Yyyyyyyyy
%
%   Chapter 2: Business Background
%
%   Change Control:
%   When     Who   Ver  What
%   -------  ----  ---  --------------------------------------------------------------
%   11Feb11  AB    0.1  Begun 
%

\chapter{Business Background}\label{C.Business.Background}

\section{Introduction}\label{S.intro2}
KPMG is one of the most renowned Big Four auditors and provides tax, audit, advisory and consultancy services to various clients. Information Risk Management is the service line of the organization that provides information systems security assurance while minimizing risks and frauds. For accuracy of financial reports, IT organizations depend on an effective audit. KPMG's IRM audit team works with clients and auditors to assist them to obtain their desired results; by assuring customers how IT functions are efficiently controlled and by ensuring auditors that their work is efficient and accurate within the guidelines. IRM audit team supports audit planning process and fraud risk assessment to monitor IT risks; supervises processes for a particular industry; supports auditors; assesses application controls design; supports testing phase of the whole audit process. Benefits of the services provided by IRM audit team are efficient and effective audits, impactful audit decisions and opinions, precise identification of business risks and issues reporting to senior management and audit committee.\\

\section{Business Contribution}
There has been a rapid loan growth since last few decades, which led to aggressive lending(weak controls and lenient standards). This increased lending can come from a volatile source. Auditing loan portfolios are imperative to make sure safety and compliance with regulatory requirements.The objective of auditing is to find errors and issues and take appropriate corrective measures or actions. Auditing of residential loan portfolios can alert users and banks about the deviations in prescribed policies of credit risks and therefore maintains sustainability and profits of banks. As mentioned in chapter \ref{C.intro} since the Irish property bubble in 2007-2010, the focus has been increased on the performance of loan portfolios especially in residential sector to achieve:
\begin{itemize}
\item Interactive way to identify patterns in datasets to drill down into problem areas
\item Well timed potential issues indicators that adhere to provisions of audit processes and assessment of residential loans
\item Better and greater coverage of problem areas and increased focus on judgemental loan applications
\item Integration of useful and relevant market data and economic indicators for enhanced loan assessment
\end{itemize}

There has been a significant improvement in technology that helps in analyzing data interactively and graphically. Growth in financial services has led to increase in accuracy of loan data and better availability of external data sources. This practicum will bring together such information in an interactive way to enhance credit analysis, audit and assessment of residential loan portfolios to reduce the cost of credit analysis, enable faster credit decisions, close monitoring of accounts and prioritize collections.
 
%   MSc Business Analytics Dissertation
%
%   Title:     Aaa Bbbbbbb Cccccccccc
%   Author(s): Xxxxxx Xxxxxxxxx and Yyy Yyyyyyyyy
%
%   Chapter 3: Literature Review
%
%   Change Control:
%   When     Who   Ver  What
%   -------  ----  ---  --------------------------------------------------------------
%   11Feb11  AB    0.1  Begun 
%

\chapter{Literature Review}\label{C.LitReview}
\section{Introduction}\label{S.intro3}

In recent years, purchasing capability of an \textbf{economy} has increased due to improvement in their finances, and employment levels. Ranging from buying small household items to\textbf{ expensive items such as a house, a car or an office}. To buy a house or a car, one needs to have a large amount of money available to him; that is not necessarily possible most of the time. \\

\textbf{Start with an exmaple to explain what is loan}. There are certain critical circumstances that can occur anytime, where one may need a certain amount of cash. So one may need to borrow a generous amount of money from some other entity which is called a loan. A loan is lending a sum of money from one entity to another that involves repayment of the amount in near future. Lent amount is called principal amount and amount to be repaid is a summation of principal amount and an interest amount or other charges. It is not as easy as it sounds like, there are certain terms need to be agreed upon by each entity before exchange of the money. A loan can be for an amount taken at one time or can be taken in instalments {Partial Payments]. A loan can be provided by banks, corporations and financial institutions. Banks and financial institutions provide various types of loans as per the need of an applicant, such as personal loans, home loans, business loans, credit card loans and cash advances. There are times when the borrowing amount is very large and banks cannot provide the loan based on verbal agreement, they need to ensure that if an applicant is not able to repay the loan then they need to have a source to recover the lent amount. So, in this case, an applicant needs to apply for a mortgage with the bank.\\

A mortgage or collateral is an instrument that applicant has to pay back with predefined series of payments to the bank and financial institutions. Over a duration of time, an applicant needs to repay the loan inclusive of interest amount in order to free his/her mortgage. In case, if an applicant is not able to repay the loan within predetermined time, then the bank can recover their money by selling or putting it for auction the mortgage. The most common type of mortgage is residential mortgages were applicant gives his/her house to banks and in a case of no repayment then a bank will claim the house to recover the balance amount of the loan. This will give a bank a security that their lent amount is not at risk and over the years they will get back their lent money one way or the other. Mortgages come in various different forms. Most commonly used mortgage types are Fixed Rate Mortgage where applicant repays the loan amount on a fixed rate throughout the period determined and Adjustable Rate Mortgage where interest rate varies as per the changes in market interest rates. Our work is based on analysis of residential mortgages with varied interest types which will be discussed in later sections.\\

Put \textbf{Photo of loan application process folw chart}

Before analysing data based on residential mortgages, one needs to understand the process of giving a loan. Depending upon the requirement an applicant applies for a loan by filling an application form with all the necessary details required by the bank. Bank officials then analyse the application and may ask an applicant for additional information; after evaluation, bank approves or disapproves the loan. Next, borrower and bank sign an agreement that states all the terms and conditions of the loan including determined interest rate and type of mortgage. Lastly, loan amount will disburse and borrower will start repaying the instalments that constitute principal amount and interest amount for predetermined period of time.\\

And, the major question is how do banks decide whether to give a loan or not? This question is of major concern as bank's cash flow highly depends on timely repayment of the loan. Every bank does not have the same procedure but majority of the loan review process is same. Following are few characteristics that bank officials will concentrate while evaluating a loan application:
\begin{enumerate}
	\item Credit history of applicant
	\item Loan to Value ratio
	\item Employment history
	\item Character assessment of applicant
	\item Evaluation of collateral
	\item Financial statements such as bank history, cash flow, etc. 
\end{enumerate}

\section{What is Credit Scoring?}\label{C.risk}

One of the most important questions of borrowing and lending process of loan is How do banks make sure whether to give a loan to a borrower or not? Banks do credit evaluation of an application to make credit management decisions. Officials collect, analyze and classify credit variables and elements to reach credit decisions. Credit evaluation determines the quality of bank. A process of evaluating customer's bad credit risk is called credit scoring. Since ages, there has been various definitions of credit scoring; \citet{hand1998consumer} stated that credit scoring is a process of measuring customer's creditworthiness. \citep{anderson2007credit} segregated credit scoring into two components : credit that means you can purchase now and repay the amount later; and, scoring means ranking based on predefined set of qualities to differentiate amongst cases in order to achieve credit decisions. On the other hand, \citet{gup2005commercial} stated that process of credit scoring uses statistical approqaches to determine whether a borrower will default in future or not. Similarly, \citet{beynon2005optimizing} said, credit scoring is a statistical model that convert relevant credit data into numerical data that support credit decisions. \citep{gup2005commercial} Credit scoring techniques have been widely used to access commercial loans, businesses, real estate industry and residential mortgages. \citep{thomas2002credit} Credit scoring is a technqiue that decide whether an applicant will get a credit, what will the process of getting credit and how will the strategies enhance borrower's profitability. Credit scoring models are very popular from last 10 decades that has made evaluation of consumer credit easy and reliable. 

\subsection{Traditional Judging System and Credit Scoring}\label{C.risk2}
Main objective of credit evaluation process is to compare and contrast characteristics of an applicant with other previous applicants who have repaid the loan amount. Bank will check applicant's profile with earlier applicants, if profile is very much similar, then they will check if applicant has repaid the loan on time. If applicant did not default then loan can be granted, if not then loan application will be rejected. \citep{crook1996credit} stated that there are two techniques for credit evaluation : Credit Scoring and Officials Subjective Assessment. Traditional judgement assessment method is totally dependent on evaluator's experience and knowledge (\citep{sullivan1981consumer} and \citep{bailey2004consumer}). Subjective assessment is subjective and inconsistent but on the other hand it can be successful, creditor's past experience can be qualitative that helps in taking successful credit decisions.\\\\
While in credit scring method, creditors use their past experience and historical information of the loan applications to form an evaluation model to determine creditworthiness. Credit scoring methos is consistent and self operated that includes quantitative measurements of applicant's credit score subjected to predictor variables such as employment duration or credit history. Also, credit scoring method provides an advantage to bank to keep their good credit customers intact and to improve customer service. Consequently, this process has been crticised because data that has been used consists of some assumptions to statistically evolve model. 

\subsection{Advantages and Disadvantages of Credit Scoring}


Used to reduce the burden of bad debt on banks similar to bubble market crisi

\section{Analysis and assessment of credit}\label{tech.crisk}

Importance of assessing credit worthiness has been increased since, the property crash in 2008. Banks and Financial instituions making efforts to enhace tranditional credit scoring mechanisms by incorporating latest technology and tools. Not only avaiablity data about customer but also rapid development in machine learning and analytics providing a foundation stone to banks. \\

Traditional credit scoring process with random selection of good and bad portfolio from creditors file around 50 - 300 \citet{capon1982credit} chartartestics points from loan portfolios to build a essential subset to perform statastical analysis .

\citep{10.2307/2983268}, Mentioned about three commonly used approaches used for selecting characterstics out of avialble data: Expert Knowledge, Stepwise Statstical procedure and evaulating individual characterstics. Subject Matter Expert(SME) \\


An applicant credit score is generated using credit rating system based on various charterstics points. Thereafter credit score is used depending on the usage of system. There are single cut-off and two cut-off stages in deciding application decision. In single cut-off, credit is granted if applicant score is higher than cut-off; otherwise credit is denied. Some instituations incorportae two stage cut-offs, in this system if credit score is higher than upper cut-off then credit is granted straighted and denied if score is lower thant lower cut-off. If score is between upper and lower cut-off then applicant credit history is pulled to calculate further scoring point and added to credit score. If new total score is higher than upper cut-off then credit is granted else denied.\\

Banks and financial instution sets their own cut-off for credit score based on the probablities of each applicant ability to repay or nonpayment of credit amount.\\

Adding flow chart of Evaulation Process and Pricing
\begin{figure}
\includegraphics[width=\textwidth]{creditscoreflow.jpg}
\end{figure}

However, Credit Risk has recevied a lot of critisim as well from Academics and Researchers. \citep{al2002credit} has questioned about optimal method to evaulate customers? What are key variables or data points which an analyst must consider while evaulating customer applications? On what basis one can classify an applicant as good or bad?\\


However, apart from above questions following can be useful when building a new credit scoring system. One should evaluate statistical techniques or algorithm by its accuracy to correctly classify historical portfolios into good or bad credit from creditors file. Also, Banks and Financial institution's identified factors that can influence the prediction of credit and loan quality by gathering all possible information from customer applications form, bank transactions history and previous credit history. Credit Analysts analysis of all these information to decide what all variables or characteristics to be included in final the credit model.\\


One of the principal objectives of credit scoring system is to assist Banks and Financial Institutions to streamline their credit management procedure and policy that will enable analysts with an efficient tool which will provide fast and accurate analysis of credit.On the longer run, such tool helps banks to avoid bad credit and scale up bank revenues and profit by selling more financial products to customers.\\


Diffrrerent Technology in Credit Risk:\\\\

\textbf{Linear Regression} allows one to build to simple model using a dependent and two or more predictor data points, and it is being used in credit scoring models as the two class problems can be represented using a dummy variable \citep{lee2005two}. A Poisson regression can be used to classify cases where customer tends to partial repayments, and these payments can represent as a Poisson count in the model. Credit analysts can promptly analyse using linear regression credit model to investigate customer factor such as past payments record, credit guarantees and default, etc. against a predefined cut-off credit score. If new applicant credit score is higher than cut-off score, then credit is granted.\\

\textbf{Discriminant analysis} is a statistical analytics technique which most commonly used by research to analysis when two or more dependent variable is categorical, and the independent variable is interval scaled. Multiple Discriminant Analysis(MDA) utilised in various studies and business verticles for the variety of applications since its inception in 1930's \citep{fisher1936use}. \citet{durand1941risk} used the Discriminant analysis for modelling a scoring system that gives a prediction about loan repayment. Many researchers agreed that the MDA is the best use to classify a group of categorical variables into two or more predictor or classes. For example, Credit Analyst can build a scoring system using MDA to categorised a new loan application into Default or Non-Default category, and this will help banks to avoid those applicants who have potential to default in repayment sooner or later.\citet{altman1968financial} used MDA by developing a scoring model based on five financial ratios by analysing financial statements to select eight variables for predicting financial bankruptcy in Corporates. \citet{eisenbeis1978problems} noted the problem associated with Discriminant Analysis such as reduction in dimensionality, improper estimation of classification error, using linear functions instead of quadratic functions, etc. Despite these limitations in MDA, it is still one of the techniques which are often used by credit analyst in building credit scoring system \citep{zhou2016research,liberati2017advances}.\\


\textbf{Logistic Regression} has resemblance with Linear regression and it is also most commonly used statsical technique for building credit scoring system. Dichtomous nature of logistic regression outcome probablity (good credit or bad credit) makes it different from linear regression. \citep{hosmer1989best}. By using two or more independent variables, one can build the simple logistic regression model. However, logistic regressions with more than one independent variables use the maximum likelihood method to build credit scoring model.\citep{altland1999regression}. Logistic regression has been widely used in building credit scoring system in financial domain ( see for example: \citep{nie2011credit,abdou2008neural,bensic2005modelling,joanes1993reject}) \\


Decision trees

fdgdfh

Expert systems



Neural networks


Genetic programming was introduced by \citet{koza1992evolution} 



























Credit analysis and assessment is very important for banks and financial instituions to evaluate the credit worthiness of an applicant or a borrower. Banks implements various factors while assessing credit risk; such as credit rating, loan to value ratio, probability of default, etc.; that leads to derivation of credit risk rating. Variety of financial techniques have been used by the officials to analyse credit risk. 

\subsection{Examples of citations}\label{SS.citations}

In \citep{Atiyah:1961,Atiyah:1966a,Atiyah:1966b}, Atiyah builds on the work of \citet{Adams:1962} to develop 
the foundations of topological $K$-theory.  \citet{LewMcG:2000} and \citet{McG:2002} later extend parts of 
this to a previously unexplored algebraic setting.

%   MSc Business Analytics Dissertation
%
%   Title:     Aaa Bbbbbbb Cccccccccc
%   Author(s): Xxxxxx Xxxxxxxxx and Yyy Yyyyyyyyy
%
%   Chapter 4: Methodology
%
%   Change Control:
%   When     Who   Ver  What
%   -------  ----  ---  --------------------------------------------------------------
%   11Feb11  AB    0.1  Begun 
%

\chapter{Methodology}\label{C.Methodology}

\section{Overview}\label{S.Ch4.opening}
To assist financial auditor or stakeholder at financial institutions and banks, and to identify such loan portfolio which may default in future based on the geospatial information and financial data. This research work followed the KDD process which involves characteristics variables selection, perform data restructuring, data transformation and data mining for the deployment of a predictive model using visual analytics tools such as Tableau, QlikView, etc.

\textbf{Software \& Tools used:}

Following is the list of tools and software that has been used while working on this project:
\begin{description}
  \item[Data Processing:] MS Excel 2017 and Alteryx Desginer 11.0
  \item[Version Control:] Github (github.com)
  \item[Dashboard:] Tableau Professional 10.2 and R Stuio 1.0.36
  \item[Data Storage:] Github Pages (https://pages.github.com/) and Google Drive
\end{description}
\textbf{R Packages used:}
\begin{description}
  \item[Packages required Logistic Regression Model:] Following packages used to building simple regression and logistic regression based model for predicting the good or bad loan portfolio: glm() with class set to "binomial" for Logistic Regression and "log" for Poisson regression, ROSE, ROCR, Dplyr, maps, ggplot2
  \item[Decession Tree:] Following r-packages used for building a predictive model based on decision tree: caret, rpart, rattle, ROSE, ROCR, RColorBrewer, party, partykit
\end{description}
\textbf{R Shiny:} R Shiny packages for building interactive dashboards: leaflet, maps, ggmap, gridExtra, htmlwidgets, reshape2. To deploy predictive model on Tableau to build dynamic and easy to use dashboard R Server used


One may replicate our work on his/her computer having minimum hardware specifications outlined here. This research work carried on following machines. 

% \usepackage{booktabs}
\begin{table}[!htb]
\centering
\caption{System configurations used to carry out this research}
\label{osc4}
\begin{tabular}{|p{3cm}|p{5cm}|p{5cm}|}
\toprule
\textbf{Specification}    & \textbf{System 1 - Lenovo Yoga 500}   & \textbf{System 2 - Dell Inspiron 15} \\ \midrule
\textbf{Operating System} & Windows 10 Professional               & Windows 7 Professional               \\
\textbf{Processor}        & Intel(R) Core(TM) i3-5005CU @ 2.00GHz & Intel(R) Core(TM) i3-3217U @ 1.80GHz \\
\textbf{RAM}              & 4.00 GB                               & 4.00 GB                              \\
\textbf{System Type}      & 64-bit OS, x64-Based Processor        & 32 -bit Operating System             \\ \bottomrule
\end{tabular}
\end{table}


\section{Data Processing \& Analysis}\label{ch4.3}

\subsection{Overview}
One requires the accessibility to the right set of data, and information on which statistical and modelling techniques can be applied to start any data oriented research in analytics domain, KPMG, Ireland provided data set. This data set contains historical data of various loan portfolios that maintained by each branch of banks or financial institutions. Also, this dataset has geospatial information about credit account along with their transactional history of previous loans. Credit scoring model requires being trained with a correct set of characteristics variables to provide the prediction with high accuracy.\\

This project has been carried out in four stages as outlined below:
\begin{itemize}
  \item Data Selection \& Processing
  \item Model Design \& Implementation
  \item Testing \& Model Results
  \item Deployment \& Visualizations
\end{itemize}

\subsection{Data Set}
Dataset format: .xlsx\\
Number of attributes: 35\\
Total number of records: 237,390\\
All the variables and attributes have been carefully studied and analysed to decide what key factors will be used to develop the model. Based on the availability of RAM on the current system, it was decided to build a model on selected characteristics variables. One may train the model with all possible variables as well if system hardware allows. Below is the list of variables in original dataset:

\begin{verbatim}

[1] "ContractRef"           "LoanBalance"    "InterestType"    
[4] "ProbationaryLoans"     "MortgageType"   "NewLoan"         "NIM"  
[8] "DefaultedLoans"        "CreditRating"   "InterestIncome"  "LTV"               
[12] "LTVCategory"          "MortgageYears"  "PropertyValue"             
[15] "MaturityDate"         "BookingDate"    "LastValuationDate"      
[18] "County"               "Branch"         "Address"   
[21] "Town"                 "InArrears"      "AddressLongitude" 
[24] "AddressLatitude"      "DaysInArrears"  "ArrearsCategory"
[27] "HousePriceMovement"   "ValueInArrears" "ValuationAgeYears"
[30] "UpdatedPropertyValue" "LTVUpdated"     "LTVCategoryUpdated"
[33] "CreditRatingMovement" "InterestRate"   "AnnualPYMT"

\end{verbatim}

Below is the comprehensive list of all variables that have been chosen for the model creation:
\begin{description}
  \item[ContractRef]: Unique reference number assigned to each portfolio
  \item[InterestType]: There are three types of interest rate: Fixed, Tracker and Variable
  \item[MortgageType]: Whether property is bought for "buy-to-let" or "owner occpied"
  \item[NewLoan]: Is portfolio is new or existing?
  \item[ProbationaryLoans]: Has loan been taken on probation?
  \item[DefaultedLoans]: Classify if the loan has defaulted in the past
  \item[LTVCategory]: 5 Level categorized pre-assigned to each loan account
  \item[CreditRating]: Each account is rated from 1-5 scale on the basis of credit union policy
  \item[MortgageYears]: How many years mortgage has been taken for?
  \item[CreditRatingMovement]: Percentage that indicates how credit rating has moved from previous value for an application
  \item[LTV]: Ratio of applied loan amount to property evaluation value 
  \item[LoanBalance]: How much loan amount is left to repay?
  \item[InterestIncome]: How much interest amount bank is earning?
  \item[PropertyValue]: Recent property evaluation amount
  \item[AnnualPYMT]: How much amount is getting repaid to the bank by the applicant annually?
  \item[AddressLatitude]: Latitude value of the house on map
  \item[AddressLongitude]: Longiitude value of the house on map
  \item[County]: Name of the county where house is located
  \item[InArrears]: Any amount that has not been paid earlier on time
  \item[ArrearsCategory]: Category that defines duration of Arrears such as more than 90 days
\end{description}

\textbf{Structure of the Data}
\begin{verbatim}
Classes ‘tbl_df’, ‘tbl’ and 'data.frame':    36696 obs. of  20 variables:
 $ ContractRef         : chr  "00000CONTR00111034" "00000CONTR00146183"
    "00000CONTR00175040" "00000CONTR00171901" ...
 $ InterestType        : Factor w/ 3 levels "Fixed","Tracker",..:
    2 3 2 1 2 3 2 2 2 3 ...
 $ MortgageType        : Factor w/ 2 levels "Buy to Let",
     "Owner Occupied": 1 2 2 2 2 2 2 2 2 2 ...
 $ NewLoan             : Factor w/ 2 levels "No","Yes":
    1 1 1 1 1 1 1 1 2 1 ...
 $ ProbationaryLoans   : Factor w/ 2 levels "No","Yes": 
    2 1 1 1 1 1 1 1 1 1 ...
 $ DefaultedLoans      : Factor w/ 2 levels "No","Yes":
    2 2 2 2 2 2 2 2 2 2 ...
 $ LTVCategory         : Factor w/ 11 levels "> 100%","0 to 10%",..:
    11 8 11 5 3 5 9 11 9 9 ...
 $ CreditRating        : Factor w/ 5 levels "1","2","3","4",..:
    4 2 4 4 3 2 3 3 4 2 ...
 $ MortgageYears       : int  31 30 30 29 29 32 28 35 29 31 ...
 $ CreditRatingMovement: int  3 0 0 0 2 -3 0 0 0 0 ...
 $ LTV                 : num  0.983 0.65 0.93 0.368 0.167 ...
 $ LoanBalance         : num [1:36696, 1] -0.647 -0.297 1.418
    -0.986 -1.32 ...
  ..- attr(*, "scaled:center")= num -1.05e-17
  ..- attr(*, "scaled:scale")= num 1
 $ InterestIncome      : num [1:36696, 1] -0.71 -0.132 1.53 
    -0.203 -1.1 ...
  ..- attr(*, "scaled:center")= num -2.14e-17
  ..- attr(*, "scaled:scale")= num 1
 $ PropertyValue       : num [1:36696, 1] -1.3 -0.311 0.779
    -0.511 -0.205 ...
  ..- attr(*, "scaled:center")= num -2.51e-17
  ..- attr(*, "scaled:scale")= num 1
 $ AnnualPYMT          : num [1:36696, 1] -1.2756 -0.2211 
    0.9057 -0.3651 ...
  ..- attr(*, "scaled:center")= num 9.48e-18
  ..- attr(*, "scaled:scale")= num 1
 $ AddressLatitude     : num  52.4 53.3 52.8 53.7 53.4 ...
 $ AddressLongitude    : num  -7.7 -6.27 -6.74 -6.68 -6.21 ...
 $ InArrears           : Factor w/ 2 levels "No","Yes":
    1 1 1 1 1 1 1 2 1 2 ...
 $ County              : Factor w/ 26 levels "Carlow","Cavan",..: 
      22 6 1 17 6 9 16 22 6 6 ...
 $ ArrearsCategory     : chr  "0" "0" "0" "0" ...
\end{verbatim}

\section{Implementation}\label{}

\begin{figure}
\centering
\includegraphics[width=\textwidth]{ETL.jpg}
\caption{ETL \& Data Model Architecture}{\textbf{Source}: Designed using MS Office}
\end{figure}


\subsection{Data Extraction} Prior building the predictive model in R one, need to process and analyse the data. The primary objective is to identify any outliers and to normalise the available data set. \citet{sola1997importance} observed that un-normalised data tends to increase square mean error and then deviate the model prediction. Therefore, it is important to treat data and normalised it's all variables so that model works with high precision and accuracy. One can also do data pre-processing using R as well, but Alteryx provides graphical user interface for selecting features and settings that makes whole data processing phase easy and fast\\

\textbf{Alteryx Desginer} tool allow one to build workflow to prepare data from multiple data sources on the go and by using features such as 'Select', 'Random Sample', 'Transform' and 'Output' one can easily prepare data for the predictive model \citep{dinsmore2016self}. Alteryx can process a large amount of data set and optimised it to be ready for data modelling in R.

\begin{center}
\begin{figure}[ht]
\includegraphics[width=\textwidth]{dataprocess.png}
\centering
\caption{Data Processing using Alteryx}{\textbf{Source:} Designed using in Alteryx Designer v11}
\label{fig:dataprocessing}
\end{figure}
\end{center}

In fig.\ref{fig:dataprocessing}, raw data has been read using \emph{Input tool}, then null values, white spaces etc removed using \emph{Cleansing Tool} and variables selection has been done using \emph{Select Tool}. To create train data set and test data set \emph{Random Sample \% tool}, which allows generating sample datasets.\\

\subsection{Data Transformation}
In Alteryx, there is no provision to normalize data. Processed data from Alteryx is loaded into \textbf{R Studio} for data normalization or scaling using in built functions such as scale($<variable>$) and log($<variable>$) on $LoanBalance$, $PropertyValue$, $InterestIncome$ and $AnnualPYMT$ as these variables are crucial paramters for credit scoring to make unbaised prediction model.\\

\textbf{R Studio}: Data from Alteryx is loaded to R Studio for the development of prediction model. R is used to identify patterns or correlation in variables using \emph{ggplot2}, \emph{plot.ly}, \emph{leaflets}. Two predictive models have developed based Logistic Regression and Decision Tree algorithms and both models performance evaluated concerning accuracy. Trained model is saved on the hard drive and loaded in Tableau, and with the help of R Server, Tableau allows the user to build dynamic visualizations. In Tableau, calculated fields can dynamically invokes R engine to perform calculations and then R results from output values back to Tableau, so that visualisations can be designed.\\


\subsection{Data Loading}\label{tableau}
\textbf{Integration of R in Tableau}: Processed and transformed data is loaded into Tableau for building business dashboards. Credit analyst or auditors will use the dashboard to identify locations where the most number of loan default happenings or identify those portfolios which have provided incorrect information, etc. business decisions can be made with the help of credit scoring dashboard.\\

\textbf{Installtion of R Server:} Local instance of R Server is deployed by installing \emph{Rserve} package from R console. To invoke R Server with following command:
      \begin{verbatim}
      install.packages("Rserve")
      library(Rserve)
      Rserve()
      \end{verbatim}

\textbf{Setting in Tableau}:\\

In Tableau, go to \emph{Settings and Performance} under \emph{Help} menu and then select \emph{Manage External Service Connection}. Following settings are required to connect with R server:
      \begin{verbatim}
      Server: "localhost" or "127.0.0.1
      Port: 6311
      \end{verbatim}

R scripts are written in calculated fields of Tableau to make calls to R using in built functions in Tableau such as \emph{SCRIPT\_STR} and \emph{SCRIPT\_REAL}



\section{Predictive Model}

\subsection{Overview}
\cite{shmueli2011predictive}, define predictive analytics as the process of building statistical models using data mining algorithm with an objective to predict the outcome on future data set. A model is evaluated based on its predictive power or accuracy. As discussed in section \ref{c3.tech}, Logistic regression and Decision Tree are most commonly algorithms for building predictive models for credit scoring. Based on the requirement of predictive algorithms, data type of certain variables has been converted using below code:

\begin{verbatim}
Datav2$CreditRating <- as.factor(Datav2$CreditRating)
Datav2$InterestType <- as.factor(Datav2$InterestType)
Datav2$MortgageType <- as.factor(Datav2$MortgageType)
Datav2$NewLoan <- as.factor(Datav2$NewLoan)
Datav2$ProbationaryLoans <- as.factor(Datav2$ProbationaryLoans)
Datav2$LTVCategory <- as.factor(Datav2$LTVCategory)
Datav2$InArrears <- as.factor(Datav2$InArrears)
Datav2$County <- as.factor(Datav2$County)
Datav2$DefaultedLoans <- as.factor(Datav2$DefaultedLoans)
Datav2$LoanBalance <- scale(Datav2$LoanBalance)
Datav2$PropertyValue <- scale(Datav2$PropertyValue)
Datav2$InterestIncome <-scale(Datav2$InterestIncome)
Datav2$AnnualPYMT <-scale(Datav2$AnnualPYMT
\end{verbatim}

\subsection{Logistic Regression}

Logistic regression is the most generally used technique for credit analysis and assessment as it works on binary response variables, i.e., 0 or 1 \citep{hilbe2011logistic}. In fig. \ref{fig:logistic}, output results of standard logistics regression function lies between 0 and 1 only. In this research work output of response variable, i.e., the probability of default $p=1$ is considered as 'Yes' and $p=0$ is considered as 'No'. Probability is represented using logistic function (logit) and the probability of binary response variable based on the one, or more independent variables.

\begin{center}
\begin{figure}[!htb]
\includegraphics[width=\textwidth]{logistic.jpg}
\centering
\caption{Standard Logistic Regression}{\textbf{Source:} http://www.thefactmachine.com/wp-content/uploads/2015/03/13-Sigmoid.gif}
\label{fig:logistic}
\end{figure}
\end{center}

\subsubsection*{Model Settings:}
\textbf{Response Variable:} DefaultedLoans\\
\textbf{Family (Function):} "Binomial" (Logit)

\subsubsection*{Model Implementation Details:}
Initially, To train the model for all variables available in the dataset, but the model couldn't be trained because R engine failed to allocate 5.0GB vector space for the model. Following the line of code is used:

\begin{verbatim}
library(stats)
m2 <- glm(DefaultedLoans ~., family = "binomial", data = trainDatav2)
\end{verbatim}

Next, model is trained with selective variables set and following code is used:

\begin{verbatim}
simpleglmv2 <- glm(DefaultedLoans ~ CreditRating + InterestIncome + 
    log(PropertyValue) + log(LoanBalance) + AnnualPYMT + LTV + 
    InterestType + NewLoan + ProbationaryLoans + MortgageYears + 
    MortgageType + InArrears + County + AddressLatitude + AddressLongitude, 
    family = "binomial", data = trainDatav2)
\end{verbatim}

Trained model is used to predict output for test dataset using following code:
\begin{verbatim}
testDatav2$prediction <- predict(simpleglmv2, newdata=testDatav2,
type="response")
\end{verbatim}


\subsection{Decission Tree}

As discussed in section \ref{logit}, Decision Trees has two most commonly used algorithm for credit scoring i.e. CART and C4.5. Classification and regression trees (CART) has been implemented using rpart() package available in R to build predicive model. rpart() syntax is \begin{verbatim}
rpart(formula, data=, method=,control=)
\end{verbatim}

\begin{description}
  \item[formula =] DefaultedLoans ~ NewLoan + County + LoanBalance + PropertyValue + InterestIncome + CreditRating + AnnualPYMT + County + LTV + LTVCategory + InArrears + MortgageType + MortgageYears + AddressLatitude + AddressLongitude
  \item[data =]trainDatav2
  \item[method =] "Class"
  \item [control =] Parameters for controlling the growth of tree. \\
  \textbf{control =  rpart.control(minisplit=500,cp = 0.001))} At least 500 observations should be on a node before attempting a split and reduce the split fit factor by 0.001 before being attempted.
\end{description}

Packages such as rattle(), RColorBrewer(), etc. used to enhance the overall decision tree.

\subsubsection*{Model Implementation Details:}

\begin{verbatim}
library(rpart)
library(rattle)                    
library(rpart.plot)            
library(RColorBrewer)       
library(party)                    
library(partykit)                
library(caret)
defaultLoanTree <- rpart(DefaultedLoans ~ NewLoan + County + LoanBalance 
+ PropertyValue + InterestIncome + CreditRating + AnnualPYMT + County 
+ LTV + LTVCategory + InArrears + MortgageType + MortgageYears 
+ AddressLatitude + AddressLongitude ,method = "class",data=trainDatav2,
control =  rpart.control(minisplit=5,cp = 0.001))

save(fit, file = "Model/classificationTreeV2.rda")
print(defaultLoanTree)
prp(defaultLoanTree)
tree.1 <- defaultLoanTree
fancyRpartPlot(tree.1)
\end{verbatim}

Finally, the Model performance of logistic regression and decision tree has been evaluated based on GINI, ROC metrics.

\section{Tableau \& Dashboards} 

Tableau professional software is used to develop the business dashboard that will be utilised by end users such as credit analyst, auditors, banks officials, etc. In Tableau, CSV file connector is used to connect to the data source ( sample dataset); then it is used to prepare various graphs and geospatial dashboard. The calculated field in Tableau allows making the call to R engine directly. By using calculated field options in Tableau, the predictive model is loaded into Tableau to make direct calls to R engine. Instructions and settings mentioned in section \ref{tableau} used as is to connect Tableau with R. \\

In the dashboard, the user can select an origin city or region and distance (in miles) from that origin. Based on these inputs user will be able to take the business decision such as investigating a loan account when property value of a particular house is higher than the area average property value, or opening new branches near by to areas for which a significant number of loan applications is coming in. Following calculations are performed in Tableau calculated fields:

\textbf{Calculation for the distance from Origin city}:
\begin{verbatim}
3959 * ACOS
(
  SIN(RADIANS(LOOKUP(AVG([Address Latitude]), First()))) * 
  SIN(RADIANS(AVG([Address Latitude]))
) +
  COS(RADIANS(LOOKUP(AVG([Address Latitude]), First()))) * 
  COS(RADIANS(AVG([Address Latitude]))) 
  * COS(RADIANS(AVG([Address Longitude])) - 
  RADIANS(LOOKUP(AVG([Address Longitude]),
  First())))
)
\end{verbatim}


\textbf{Calculation script for logistic regression model in Tableau:}
\begin{verbatim}
SCRIPT_REAL('mydata <- data.frame(DefaultedLoans=.arg1, CreditRating=.arg2,
InterestIncome=.arg3, LoanBalance =.arg4, AnnualPYMT =.arg5, LTV =.arg6,
InterestType=.arg7,NewLoan=.arg8, ProbationaryLoans = .arg9, 
MortgageYears=.arg10,MortgageType=.arg11, InArrears =.arg12,County =.arg13,
AddressLatitude=.arg14, AddressLongitude=.arg15, PropertyValue=.arg16);
load("Model/simpleglmv2.rda")

prob <- predict(simpleglmv2, newdata = mydata, type = "response")',
ATTR([Defaulted Loans]),ATTR([Credit Rating]),AVG([Interest Income]),
AVG([Loan Balance]),AVG([Annual PYMT]),AVG([LTV]),ATTR([Interest Type]),
ATTR([New Loan]),ATTR([Probationary Loans]),AVG([Mortgage Years]),
ATTR([Mortgage Type]),ATTR([In Arrears]),ATTR([County]),
AVG([Address Latitude]),AVG([Address Longitude]),AVG([Property Value]))
\end{verbatim}





%   MSc Business Analytics Dissertation
%
%   Title:     Aaa Bbbbbbb Cccccccccc
%   Author(s): Xxxxxx Xxxxxxxxx and Yyy Yyyyyyyyy
%
%   Chapter 5: Results
%
%   Change Control:
%   When     Who   Ver  What
%   -------  ----  ---  --------------------------------------------------------------
%   11Feb11  AB    0.1  Begun 
%

\chapter{Results}\label{C.Results}

\section{Overview}
Model prediction accuracy of original test data set was 99.65\%, which is practically impossible. As discussed in chapter \ref{C.intro} actual data received from KPMG was made up using pre-defined formulas and rules to make it look real. Data didn't cover all possible scenario for a loan portfolio and achieving an accuracy of 99\% in credit scoring model is difficult as one needs to train model recursively with large data size covering all permutations and combinations of situations for loan default.


\section{Introduction}\label{S.intro5}

Original data set consist of 237389 observations and 35 variables, according to data 95\% loan applications will not default, and only 5\% application had chances to default. Therefore, to consider all possible scenarios data has been modified and a data subset has been generated from original dataset to carry experiments. New dataset has 36696 observations and 26 variabls. 

\section{Performance}
\textbf{Decision Tree over Logistic Regression:}

\cite{long1993comparison} studied decision tree application for classifying heart disease patient and compared the performance of decision tree with logistic regression. \cite{long1993comparison}, also noted that logistic regression model failed to consider missing data and decision tree model easily worked when data was noisy. \cite{satchidananda2006comparing}, build credit scoring model and found that decision tree produce a more precise model and good performance in comparison to logistic regression.

In this research work, decision tree performance did better against the logistic regression performance. Decision tree accuracy is 81.11\%, and logistic regression accuracy is 68.34\%. Based on the results of previous research work and after considering current experiments results on KPMG dataset, it is appropriate to build the business dashboard using decision tree model.

Another advantage of using decision tree model is that one can control the growth of decision tree using 'split' setting by doing so model performance can be optimised. On the other hand,  to train model with logistic regression, one to select the restricted number of independent variables, otherwise, the model can not be trained with many variables as vector size response variable grows exponentially.

Significant Variables in Logistic regression:
\begin{verbatim}
[1] "CreditRating"               "PropertyValue"             
 [3] "LoanBalance"                "LTV"                       
 [5] "NewLoanYes"                 "ProbationaryLoansYes"      
 [7] "MortgageTypeOwner Occupied" "CountyCavan"               
 [9] "CountyDonegal"              "CountyDublin"              

\end{verbatim}

\begin{table}[!htb]
\centering
\caption{Comparison of Logistic Regression and Decision Tree performance}
\label{table:results}
\begin{tabular}{@{}lccc@{}}
\toprule
\textbf{Model}               & \textbf{AUROC} & \textbf{KS} & \textbf{Gini} \\ \midrule
\textbf{Logistic Regression} & 68.34          & 13.53       & 36.68         \\
\textbf{Decision Tree}       & 81.11          & 60.04       & 62.22         \\ \bottomrule
\end{tabular}
\end{table}

In table. \ref{table:results}, KS is the Kolmogorov-Smirnov goodness-of-Fit Test (or KS-Test), GINI is Gini coefficient of inequality distribution of response variable and AUC is Area under ROC (Receiver Operating Chaterstics) curve.\\

\begin{figure*}[t!]
    \centering
    \begin{subfigure}
        \centering
        \includegraphics[height=1.2in]{LRrecall.png}
        \caption{Logistic Regression}
    \end{subfigure}%
    \begin{subfigure}
        \centering
        \includegraphics[height=1.2in]{DRrecall.png}
        \caption{Decision Tree}
    \end{subfigure}
    \caption{Precision vs Recall curve}
\end{figure*}


\begin{center}
\begin{figure}[!htb]
\includegraphics[width=\textwidth]{results1.png}
\centering
\caption{ROCs for logistic regression vs decision tree}{\textbf{Source:} Plotted in R Studio}
\label{fig:result}
\end{figure}
\end{center}

Receiver under curve is one of technique to estimate the performance of predictive model.




%   MSc Business Analytics Dissertation
%
%   Title:     Aaa Bbbbbbb Cccccccccc
%   Author(s): Xxxxxx Xxxxxxxxx and Yyy Yyyyyyyyy
%
%   Chapter 6: Discussion
%
%   Change Control:
%   When     Who   Ver  What
%   -------  ----  ---  --------------------------------------------------------------
%   11Feb11  AB    0.1  Begun 
%

\chapter{Discussion}\label{C.Discussion}
\section{Introduction}\label{S.Discussion.intro}
\begin{center}
\begin{figure}[!htb]
\includegraphics[scale=0.3]{crash1.png}
\centering
\caption{Irish Property Crash}{\textbf{Source:} Tableau Professional v10}
\label{fig:crash1}
\end{figure}
\end{center}
This chapter presents the detailed discussion and analysis of patterns and trends discovered during this research work. Keeping the interest of every stakeholder from bank officer to auditors, an attempt has been build simplicity in the business dashboard so that end user can use it efficiently to drive the business decision. When it was discovered that original data is not appropriate from the predictive modelling perspective, the modified data has been used throughout this research work. All analysis has been presented considering modified data set.

\section{Patterns \&  Analysis}

Since the property crash (2007-2010) average property price and the number of loan applications has been reduced as it can be seen in fig \ref{fig:crash1}. Earlier the average property was \euro 120K then it increased by 100\%.

\begin{center}
\begin{figure}[!htb]
\includegraphics[scale=0.4]{clustero.png}
\centering
\caption{County Cluster}{\textbf{Source:} Tableau Professional v10}
\label{fig:clustero}
\end{figure}
\end{center}

k-Mean algorithm is used to cluster twenty six(26) counties among four cluster based on loan balance, as Ireland property market vary a lot in county. Co. Dublin is classified into cluster \#2 and Co. Cork is classified into cluster \#4 has highest outstanding loan balance. As shown in fig. \ref{fig:cluster} 8 counties are in cluster \#1 and 16 counties with least outstanding loan balance classified into cluster \#3.

\begin{center}
\begin{figure}[!htb]
\includegraphics[scale=0.5]{cluster.png}
\centering
\caption{Clustering Results}{\textbf{Source:} Tableau Professional v10}
\label{fig:cluster}
\end{figure}
\end{center}

During data analysis phase, it was noted that the most properties has LTV(Loan-to-value) ratio between 60 - 80\%. There are few properties with LTV higher than 100\%.
In fig. \ref{fig:ltv}
\begin{center}
\begin{figure}[!htb]
\includegraphics[scale=0.5]{ltv.png}
\centering
\caption{Loan to Value histogram}{\textbf{Source:} R Studio ggplot()}
\label{fig:ltv}
\end{figure}
\end{center}

\begin{center}
\begin{figure}[!htb]
\includegraphics[scale=.4]{countynumber.png}
\centering
\caption{Number of account county wise}{\textbf{Source:} Tableau Pro}
\label{fig:tableaucounty}
\end{figure}
\end{center}

A total number of accounts in the normalized data set was 273,000 , and most numbers of default and loan account were from County Dublin and Cork as seen in fig. \ref{fig:tableaucounty}.

\section{Statistical Analysis}

Descriptive analysis has been performed using SPSS tool to get better understanding of data and take required steps for data preprocessing.

\begin{center}
\begin{figure}[!htb]
\includegraphics[scale=0.7]{stats1.png}
\centering
\caption{Analysis of Significant variables - 1}{\textbf{Source:} SPSS}
\label{fig:stats1}
\end{figure}
\end{center}


\begin{center}
\begin{figure}[!htb]
\includegraphics[scale=0.7]{stats2.png}
\centering
\caption{Analysis of Significant variables - 2}{\textbf{Source:} SPSS}
\label{fig:stats2}
\end{figure}
\end{center}



\section{Dashboard}
Considering the day-to-day requirement of stakeholders at KPMG, three business dashboards are designed using Tableau software. Main response to build dashboard using Tableau,as it allow the easy integrations with predictive model from R engine and provides a better means to visualize prediction and forecast.

\begin{center}
\begin{figure}[!htb]
\includegraphics[width=\textwidth]{Overview.png}
\centering
\caption{Portfolio Overview}{\textbf{Source:} Tableau Professional v10}
\label{fig:overview}
\end{figure}
\end{center}


\textbf{Portfolio Overview}\\
Portfolio overview dashboard is built to allow auditors and credit analysts to analyse overall existing loan portfolio registered under a bank. This dashboard allows the user to perform a detailed analysis by selecting various combinations of variables from filters such as:
\begin{itemize}
\item Loan Start Date: User can view selective number of loan account based on the start date of a account
\item Is in Arrears?: Does a loan account has any outstanding repayment in last one year?
\item Mortgage Type: For what purpose mortgage has been buy-to-let or owner occupied?
\item Is probationary Loan?: Has the loan account been converted to probationary loan
\end{itemize}

The geospatial map allows analyst to view loan balance for each county, along with a time-series analysis of property price and number of accounts for past three decades. Using interest type vs mortgage type user can identify which type of interest is giving more income to banks.

\begin{center}
\begin{figure}[!htb]
\includegraphics[width=\textwidth]{Analysis.png}
\centering
\caption{Portfolio Analysis}{\textbf{Source:} Tableau Professional v10}
\label{fig:overview}
\end{figure}
\end{center}
\textbf{Portfolio Analysis}\\
Portfolio analysis dashboard allow user to view loan account movement from one LTV (loan to value) category to other, along with unemployment rate and average property sale price in a town. User can select range of property from map using a distance(radius in kms as in fig \ref{fig:dist}) to compare trends in selected neighbourhood.

\begin{center}
\begin{figure}[!htb]
\includegraphics[width=\textwidth]{dist.png}
\centering
\caption{Property Selection using distance}{\textbf{Source:} Tableau Professional v10}
\label{fig:dist}
\end{figure}
\end{center}

By using open source library of R Shiny, a dynamic dashboard has been build which allow user to view property based on clustring. Also user can view street level statstical metrics.

\begin{center}
\begin{figure}[!htb]
\includegraphics[width=\textwidth]{rmap.png}
\centering
\caption{Dashboard Build using R Shinny}{\textbf{Source:} R Studio}
\label{fig:rmap}
\end{figure}
\end{center}



\section{Success}\label{S.succes}
To measure the performance of this work, a working business dashboard has been given to stakeholder to use it and provide their feedback. Based on their feedback and suggestions dashboard features has been implemented accordingly. The user was given two dashboard design and asked to recommend best one with possible changes and suggestions to further improve usability.

\begin{itemize}
\item \textbf{47\%}Auditors feel this dashboard is extremely useful
\item \textbf{70\%}Feel that geospatial techniques have enhanced credit assessment
\item \textbf{60\%}Likely to recommend this dashboard to colleagues
\end{itemize}
%   MSc Business Analytics Dissertation
%
%   Title:     Aaa Bbbbbbb Cccccccccc
%   Author(s): Xxxxxx Xxxxxxxxx and Yyy Yyyyyyyyy
%
%   Chapter 7: Conclusions and Future Research
%
%   Change Control:
%   When     Who   Ver  What
%   -------  ----  ---  --------------------------------------------------------------
%   11Feb11  AB    0.1  Begun 
%

\chapter{Conclusions and Future Research}\label{C.Conclusions.Future.research}

\section{Integration with real data}

Credit scoring is a senstive and important component of any bank and financial institutions. The outcome of this work presents a predictive model that connects with a business dashboard. One can integrate the day to day financial data from a bank with this dashboard. This work can be improvised with the help of real banking data so that predictive model can be trained efficiently.
\begin{description}
\item[Dashboard:] Real transactional data can improve the performance of Tableau dashboard which acts as decision support system 
\item[Geospatial Data:] Credit assessment can be enhanced if it includes information such as house coordinates
\item[External Factors:] The model can perform better when trained with large number of external factors such as medical information, average salary in neighbourhood

\end{description}

\section{Future Work}
\begin{center}
\begin{figure}[!htb]
\includegraphics[width=\textwidth]{future.png}
\centering
\caption{System Architecture}{\textbf{Source:} MS Powerpoint}
\label{fig:future}
\end{figure}
\end{center}
\section{Conclusion}
The objective of this project was to build an interactive and efficient dashboard that supports credit analysis and assessment of residential loan portfolios with the help of geospatial methods. This practicum was stemmed on aggregation of loan portfolio data and financial services data that adheres to credit assessment policies and macroeconomic performance indicators. The purpose of developing predictive models for calculating the probability of default using logistic regression and decision trees was successfully achieved.  Initial review of the literature revealed that majority of the researchers believe models developed using logistic regression show better performance compared to decision trees regarding accuracy and area under ROC, but few have concluded the opposite. This practicum falls into the category of those researchers, who have stated decision trees give better performance than logistic regression based on KS, GINI and ROC statistics.  Although, because of the limitation of data, it cannot be said that the developed predictive model will show similar results when connected to real life dataset. There are possibilities that logistic regression can give better performance compared to decision trees.



%%   These next two items are more or less the same (epilogue is just the Greek for afterword!)
%%   They are unlikely to be long enough to justify their own file(s) to \include

%%   Epilogue (if any)
%%   Afterword (if any)

%% Back matter (loose ends: continue arabic page numbering)
\backmatter

%%   Appendices, if any: there will almost certainly be one or more.
%%   This is where you put material which the reader may like to refer to but which might break
%%   the train of thought in the thesis proper.  Large tables, etc might go here.
%%   You may also include program code if not too long: say a few important chunks.  But the
%%   majority of this should just be put on the accompanying CD/DVD.
%%   \include appendix/ces as it/they will be big enough to justify its/their own ``chapter(s)''
\cleardoublepage
\appendix
\addappheadtotoc         % adds a separating entry to the TOC, saying Appendices
\noappendicestocpagenum  % ensures there is no page number after that TOC separating entry
% It may happen that the first appendix (e.g., Detailed Tables) appears before the "Appendices" line
% in the table of contents.  If so, edit thesis.toc and move the Appendices line there above the first appendix,
% then re-run LaTeX to get the thesis.dvi file right.  You may have to repeatedly do this every time you LaTeX!
\cleardoublepage
%   MSc Business Analytics Dissertation
%
%   Title:     Aaa Bbbbbbb Cccccccccc
%   Author(s): Xxxxxx Xxxxxxxxx and Yyy Yyyyyyyyy
%
%   Appendix 1: long tables
%
%   Change Control:
%   When     Who   Ver  What
%   -------  ----  ---  --------------------------------------------------------------
%   11Feb11  AB    0.1  Begun 
%

\chapter{Detailed tables}\label{C.Appendix1}

Xyz

%   MSc Business Analytics Dissertation
%
%   Title:     Aaa Bbbbbbb Cccccccccc
%   Author(s): Xxxxxx Xxxxxxxxx and Yyy Yyyyyyyyy
%
%   Appendix 2: program code
%
%   Change Control:
%   When     Who   Ver  What
%   -------  ----  ---  --------------------------------------------------------------
%   11Feb11  AB    0.1  Begun 
%

\chapter{Program code}\label{C.Appendix2}

Xyz etc

% and so on

%%   Endnotes (if needed)

%%   Glossary of terms
%%   \include this as it may be big enough to justify its own ``chapter''
\cleardoublepage
\addcontentsline{toc}{chapter}{Glossary}
%   MSc Business Analytics Dissertation
%
%   Title:     Aaa Bbbbbbb Cccccccccc
%   Author(s): Xxxxxx Xxxxxxxxx and Yyy Yyyyyyyyy
%
%   Glossary
%
%   Change Control:
%   When     Who   Ver  What
%   -------  ----  ---  --------------------------------------------------------------
%   11Feb11  AB    0.1  Begun 
%

\chapter*{Glossary}\label{C.Glossary}

Entries are listed in alphabetical order.  



%%   Bibliography (References)
% For this, use BibTeX (invariably comes with your LaTeX installation).  BibTeX separates the form of a
% reference from its content, just as LaTeX does for a document.  You put the content in a .bib file
% called a BibTeX database, and tell BibTeX what style to overlay on it.
\cleardoublepage
\addcontentsline{toc}{chapter}{Bibliography}
\bibliographystyle{mscBA} % the BibTeX style file to use (omit the .bst suffix)
\bibliography{thesis}     % the BibTeX database file to use (omit the .bib suffix)
%
%% In the above, mscBA.bst is a BibTeX style file I developed based on the fairly standard chicago.bst.
%%
%% Below are sample journal and book entries from a .bib file.  Ensure that the key e.g., Chomsky:1965 you use
%% in your .bib file entry is the same as you use in the \cite{Chomsky:1965} command in your LaTeX file.
%%
%% @String{MIT = {{MIT Press}}}
%% @String{MIT:adr = {{Cambridge, Massachusetts}}}
%%
%% @Article{MetropEtAl:1953,
%%  AUTHOR =       {Metropolis, N. and A.W. Rosenbluth and M.N. Rosenbluth and A.H. Teller and E. Teller},
%%  YEAR =         {1953},
%%  TITLE =        {Equations of {S}tate Calculations by Fast Computing Machines},
%%  JOURNAL =      {Journal of Chemical Physics},
%%  VOLUME =       {21},
%%  NUMBER =       {6},
%%  PAGES =        {1087--1092}
%% }
%%
%% @Book{Chomsky:1965,
%%   Author =       {N. Chomsky},
%%   title =        {Aspects of the theory of syntax},
%%   PUBLISHER =    MIT,
%%   ADDRESS =      MIT:adr,
%%   YEAR =         1965
%% }
%%
%% Some other points (see the sample bibfile entries above for illustration):
%% - Case is unimportant for a field name: DATE, daTe and date all mean the same.
%% - You separate multiple authors by 'and', not using commas etc.
%% - Some punctuation marks in the key are OK, e.g. the colon ':' as in MetropEtAl:1953
%% - BibTeX has its own internal rules on when to change the case of capitalised text (e.g. in journal article titles).
%%   You can override this by putting braces {} around text you want unchanged (e.g. the S in the article title above.
%% - You can define strings which will be expanded in the references, e.g., MIT and MIT:adr in the book above
%%   Don't put braces or quotes around these: if you do, they will not be expanded (will remain as MIT:adr, say)
%% - In your actual text, you can cite in two ways (note the position of parentheses):
%%   \citep{MetropEtAl:1953} for citation in parentheses, e.g.,
%%     "In \citep{MetropEtAl:1953} it is shown that ..." appears as "In (Metropolis et al, 1953) it is shown that ..."
%%   \citet{MetropEtAl:1953} for citation in text, e.g., where the authors are the subject of the sentence:
%%     "\citet{MetropEtAl:1953} show that ..." appears as "Metropolis et al (1953) show that ..."
%% - With either \citep or \citet you can use optional arguments such as
%%     "\citep[e.g.,][pp.~12--20]{MetropEtAl:1953}" appears as "(e.g., Metropolis et al, 1953, pp.~12--20)"

%%   List of contributors -- anyone else you feel should be credited

%%   Notation Index / List of Symbols
%%   \include this as it may be big enough to justify its own ``chapter''
\cleardoublepage
\addcontentsline{toc}{chapter}{List of Notation}
%   MSc Business Analytics Dissertation
%
%   Title:     Aaa Bbbbbbb Cccccccccc
%   Author(s): Xxxxxx Xxxxxxxxx and Yyy Yyyyyyyyy
%
%   List of Notation
%
%   Change Control:
%   When     Who   Ver  What
%   -------  ----  ---  --------------------------------------------------------------
%   11Feb11  AB    0.1  Begun 
%

\chapter*{List of Notation}\label{C.Notation}

Entries are listed in the order of appearance.  The ``Ref'' is the number of the section, 
definition, etc., in which the notation is explained.

\vspace{0.5cm}

{\renewcommand{\arraystretch}{0.9}

\begin{tabular}{llr}
\tb{Symbol}  & \tb{Description} & \tb{Ref}   \\\hline
$\FF_q $  & Finite field of $q$ elements & \ref{Th.FF.fte.field}  \\
\end{tabular}


}



%%   Index
\cleardoublepage
\addcontentsline{toc}{chapter}{Index}
\printindex

\end{document}
